\documentclass{article}
\usepackage{graphicx}
\begin{document}



\begin{large}\textbf\center{A REPORT ABOUT A SYSTEM THAT COLLECTS DATA CONCERNING CANTEENS IN MAKERERE UNIVERSITY.}\end{large}



\section{Introduction}
{This data collection concept is about finding the different canteens in Makerere university, the number of students served on each canteen, the location of the canteen and its picture.
The main objective of this project is to develop a flexible system using an Open Data Kit (ODK).This system collects data of different kinds including;audio, video, text and generates GPS coordinates automatically.The system also involves an aggregate server on which this data is uploaded and it stores this data.

The system has an xml form that is created using ODK build and it is transferred to phone. The form is then stored in ODK collect/forms folder and then he/she can enter dataabout the canteen. After this, the user is able to transfer this data to the ODK aggregate server.
}

\section{Problem statement.}
There has been increasing questions about the different canteens in Makerere university, the number of students they serve, and their locations.This leaves many important details about these canteens unknown to the university management.

\section{Problem solution}
The data collection system i have designed addresses all the above problems. It allows a user to write the name of the canteen,  the number of students served on each canteen, the location of the canteen and its picture. All these details are contained in xml form that was created using the ODK build, then this is uploaded to aggregate server that was created using the Google AppEngine via thhis url "muk-canteens-168012.appspot.com".

From this server, the university management can be able to extract information about the different canteens in the university.

\section{Objectives to the project.}
To know where students get food from.
To find out the number of students served at each canteen.
To find out where each canteen in the university is located and also record its GPS coordinates.

\section{User requirements}
The system is compatible since it can be used by all versions of smart phones.

Inorder to upload the collected data to the server, the user needs an uninterrupted internet collection.

It’s user friendly that is; it’s easy to use since the features it contains are self-explanatory and are in a simple and precise language.

The system is time saving since it needs only four fields which need input.


\includegraphics[width=3cm, height=4cm]{image4}\includegraphics[width=3cm, height=4cm]{image3}\includegraphics[width=3cm, height=4cm]{image2}\includegraphics[width=3cm, height=4cm]{image1}





 \section{The functional requirements}
The phone on which the xml form is stored needs some storage not less than 10MB.

The user needs to know the url generated by the GoogleAppEngine. This unique url will connect the user to the aggregate server on which data is uploaded.


\section{Conclusion}\label{sec:into}

Basing on the results from the field it shows that most cateens are not registered with Makerere University


\section{References}\label{sec:into}

[1] Dr.OBBO BLESSEAD PETER.
Manager at vernus interprises.


[2] Miss. Joan Kigongo
Assitant metron Africa hall




\end{document}